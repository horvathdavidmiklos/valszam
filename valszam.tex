\documentclass{article}
\usepackage{amsmath}
\usepackage{amssymb}
\usepackage{enumerate}
\usepackage[a4paper, left=2.5cm, right=2.5cm, top=1.5cm, bottom=1.5cm]{geometry}


\title{Valószínűségszámítás Feladatmegoldások}
\author{}
\date{2025/2. félév}

\begin{document}
\maketitle

\section{Valószínűségek kiszámítása (ismétlés: kombinatorika)}

\subsection*{1.3. Feladat:}
Ha egy magyarkártya-csomagból (32 lap: piros, zöld, makk, tök) visszatevéssel húzunk három lapot, akkor mi
annak a valószínűsége, hogy:

\subsubsection*{a) pontosan egy piros színű lapot húztunk?}
\begin{itemize}
    \item $A = \text{\{pontosan 1 db piros színű lapot húztunk\}}$
    \item $P(A) = ?$
    \item $ \Omega = \text{\{(P1,P1,P1), (P1,P1,P2)...\}}$
\end{itemize}
\[P(A) = \frac{\text{kedvező}}{\text{összes}} = 
    \frac{8 \cdot 24 \cdot 24 + 24 \cdot 8 \cdot 24 + 24 \cdot 24 \cdot 8}{32^3} = \\
    \frac{3}{1} \cdot \frac{8}{32}^1 \cdot \frac{24}{32}^2
\]
\textit{Megjegyzés:  $\frac{3}{1}$ : 3db-ból választunk, $\frac{8}{32}^1$ 1db piros  $\frac{24}{32}^2$ 2db nem piros}

\vspace{1em} 
\subsubsection*{b) legalább egy piros színű lapot húztunk?}
\begin{itemize}
    \item $B = \text{\{legalább 1 db piros színű lapot húztunk\}}$
\end{itemize}
\[1-P( \overline B) = 1 - \frac{24^3}{32^3}\]


\subsection*{1.4 Feladat}
Egy zsákban 10 pár cipő van. 4 db-ot kiválasztva, mi a valószínűsége, hogy van közöttük pár, ha:

\subsubsection*{a) egyformák a párok}
\begin{itemize}
 \item cipők száma: $2 \cdot 10$ azaz 20 db
\end{itemize}
\[ P( \text{lesz pár}) = 1 - P( \overline{\text{lesz pár}}) = 1 -\frac{2 \cdot \binom{10}{4}}{\binom{20}{4}} \]


\subsection*{1.5. Feladat}
 \textbf{n} dobozba véletlenszerűen helyezünk el \textbf{n} golyót úgy, hogy bármennyi golyó kerülhet az egyes dobozokba.
\begin{itemize}
    \item \textbf{n} doboz \textbf{n} golyó
   \end{itemize}
\subsubsection*{a) Mi a valószínűsége, hogy minden dobozba kerül golyó?}
\[ \text{P(minden dobozba kerül golyó)} = \frac{n!}{n^n} \]
\textit{Megjegyzés: összes lehetséges eset: $n^n$ mivel egy dobozba több golyó is lehet, kedvező eset: $n!$ mivel elősbe n-ből lehet választani, másodikba $(n-1)$ből ...}


\subsubsection*{b) Annak mi a valószínűsége, hogy pontosan egy doboz marad üresen?}
\[ \text{P(pont egy doboz üres)} = \frac{n(n-1)n!\frac{1}{2}}{n^n} \]
\textit{?????}

\section{Feltételes valószínűség és Bayes-tétel, diszkrét valószínűségi változók}
\subsection{2.12. Feladat}
Jelölje X az ötöslottón kihúzott lottószámok legkisebbikét. Adjuk meg X eloszlását!

\begin{itemize}
    \item 90 szám, 5-öt húzunk ki
    \item x := a legkisebb kihúzott szám
    \item eloszlása = ?
\end{itemize}

\[ P_k =  \mathbb{P}(x=k) = \frac{\binom{90-k}{4}}{\binom{90}{5}}\]
\textit{Megjegzés: a legkisebb szám 1-től 86ig lehet}


\subsection{2.9. Feladat}
Tegyük fel, hogy az új internet-előfizetők véletlenszerűen választott 20\%-a speciális kedvezményt kap. Mi a
valószínűsége, hogy 10 ismerősünk közül, akik most fizettek elő, legalább négyen részesülnek a kedvezményben?

\[ P_k = \mathbb{P}(x=k) = \binom{n}{k} p^k (n-p)^n-k \]
\[ \mathbb{P}(x \geq k) = \sum_{k=4}^{10} {\mathbb{P}(x=k)} =\sum_{k=4}^{10} {P_k}  \]
\[ 1 - \mathbb{P}(x \leq k) = 1 - \sum_{k=0}^{3} {P_k}  \]



\end{document}

